\documentclass[utf8,english,DIV=12,12pt]{scrartcl}
\usepackage{AnonymousPro}
\usepackage{FiraSans}
\usepackage[osf]{mathpazo}
\usepackage[T1]{fontenc}
\usepackage{inputenc}
\usepackage{babel}
\usepackage{amsmath,amssymb}
\usepackage{microtype}
\usepackage{tikz}
\usepackage{booktabs}
\usepackage{ifthen}
\colorlet{lcolor}{blue!20!black}
\colorlet{ucolor}{magenta!20!black}
\colorlet{ccolor}{green!20!black}
%\usepackage{url}
% \usepackage[language=english,%
%              backend=biber,%
% %            maxnames=3,%
% %            minnames=2,%
% %          firstinits=true,%
%             bibstyle=authoryear,%
%            citestyle=authoryear-comp,%
%                  url=false,%
%               eprint=false,%
%               dashed=false]{biblatex} % authoryear-comp

\usepackage{varioref} 
\usepackage[colorlinks=true,%
            linkcolor=lcolor,%
            urlcolor=ucolor,%
            citecolor=ccolor]{hyperref} % makes references clickable
            % links will be in specified colors. You may set them to
            % black for final printing.




\newcommand{\noise}{\ensuremath{\varkappa}}
\newcommand{\ode}{\textsc{ode}}
\newcommand{\sens}[5][j]{\ensuremath{S_{#2}(t_{#1};\rho,u_{#5})_{#3}^{~#4}}}
\newcommand{\software}[1]{\textsc{#1}}
\newcommand{\standard}[1]{\textsf{#1}}
\newcommand{\stdv}[1]{\ensuremath{\ifthenelse{\equal{#1}{g}}{\varsigma}{\sigma}}}
\newcommand{\data}[1]{\ensuremath{\ifthenelse{\equal{#1}{g}}{y}{z}}}

\frenchspacing
\setcapindent{0em}
\addtokomafont{captionlabel}{\bfseries}
\addtokomafont{pagenumber}{\sffamily}
\pagestyle{headings}

\subject{\texttt{MCMC\_CLIB}}
\title{Documentation}
\author{Andrei Kramer}
\date{\today}

\begin{document}
\maketitle
\begin{abstract}\noindent
  Here, we explain the model structure assumed by \texttt{mcmc\_clib}.
  In Section~\ref{sec:model} we define the \ode{} model, its
  input/output structure and its parameter sensitivity. We describe
  how the model is passed to the sampling software in
  Section~\ref{sec:software} and how to build a model as a shared
  library. All data sets have to be loaded in \standard{hdf5} form, with
  specific naming conventions. This is described in
  Section~\ref{sec:data}. Some of the symbols have shorthands defined
  in the text and the meaning of any variable can be overloaded by
  such a shorthand symbol.
\end{abstract}

\section{Model Specifications}
\label{sec:model}

We consider a deterministic \ode{} model and a stochastic measurement
model. System states are captured by the state variables
$x(t;\rho,u)\in\mathbb{R}^n$. The model parameters
$\rho = \exp(\theta) \in\mathbb{R}_{++}^m$ are for the most part reaction
rate coefficients and equilibrium parameters in systems biology; they
describe interactions between the model state variables and are
\emph{unknown}. If any coefficients are exactly known, they should be defined
as known constants in the model rather than parameters perhaps.

A second set of parameters $u\in\mathbb{R}^\upsilon$ describes the conditions
of an experimental setup. These parameters are considered inputs and
we assume that they can be set by the experimenter (and therefor
\emph{known}). They can be external parameters, e.g. the temperature,
or describe modifications to the system, e.g. inhibitions to some of
the interactions. A special kind of inputs are measured data sets of
compounds that the model lacks a mechanism for, so their state has to
be replayed, perhaps using interpolation, during simulations. Such an
interpolation has to be implemented by the user, in the model; the
software has no automated interpolation of input signals by itself.

Any experiment observes data $y$ that is analogous\footnote{up to
  measurement device specific scalings, offsets or other arbitrary
  constants} to the model's output function $g(x)\in\mathbb{R}^r$. The
modeling assumption is that the data can be explained by the model
output aside from measurement noise:
\begin{align}
 \data{g}_{ijk} - c_k g_i(x(t_j;\rho,u_k)) &\sim \mathcal{N}(0,\stdv{g}_{ijk}^2)& x(0;\rho,u)&=x_0\,,\\
\end{align}
where $c_k$ is a possibly unknown scaling constant that accounts for
experiments with \emph{arbitrary units}. Data from such an experiment
is called \emph{relative data}. In any case, the data is obtained at
measurement time-points $t_{jk}$ ($j=1,\dots,\nu_k$); these can be
different in each experiment, and $t_{jk} \neq t_{jk'}$ is
allowed. But we'll omit this from the notation and just use $t_j$.

There are several ways to deal with relative data. Our choice was to
consider a second level of output function $h$:
\begin{align}
  h_i(g_{ijk},g_{ij'0})&=\frac{g_i(x(t_j;\rho,u_k))}{g_i(x(t_{j'};\rho,u_0))}\,,& g_{ijk}:&=g_i(x(t_j;\rho,u_k))\,,\\
  \data{h}_{ijk} - h_i(x(t_j;\rho,u_k)) &\sim \mathcal{N}(0,\stdv{h}_{ijk}^2)& \data{h}_{ijk}&=\frac{y_{ijk}}{y_{ij'0}}\,,
\end{align}
such that scale constants $c$ always cancel; and we define the
shorthand $g_{ijk}:=g_i(x(t_j;\rho,u_k))$. Here we assume that $u_0$
determines the so called \emph{control} experiment. It is the
reference experiment that experiment $k$ is \emph{relative to}. But,
it can be any $k'$ in principle; this notation choice neatly implies
that if $k=0,\dots,n_{\text{E}}$ the \emph{number of experiments} is
still effectively $n_{\text{E}}$ (not $n_{\text{E}}+1$) as one of them
is merely a \emph{normalization} (control). The difference between $g$
and $h$ is that $g$ corresponds to exactly one model simulation, while
$h$ processes several simulations. The output $g$  is defined within the model, while $h$ is specified through data annotation.

A data set can contain several (unrelated) \emph{controls} if needed.
Experiments have an annotation to express the relationship between
them. The \emph{reference time} $\hat t=t_{j'}$ does not have to be
the same as $t_j$ and can also be defined through annotation of data
sets. In principle $h$ can also mix output functions and devide
$g_{i\cdot\cdot}$ by $g_{i'\cdot\cdot}$.

We consider \emph{different experiments} to be distinguished in
\emph{input parameters}, \emph{initial conditions}, or \emph{output
  function} (at least one of these). All applicable inputs are
enumerated as $u_k\in\mathbb{R}^\upsilon$ ($k=0,\dots,n_{\text{E}}$).

\subsection{The Mechanistic Model}
\label{sec:ODE}

The model class the software can deal with is an ordinary differential
equation (\ode) model:
\begin{align}
  \dot x &= f(t,x;\theta,u)\,,& x(t_0)&=x_0\\
\end{align}
The initial conditions $x_0$ can be part of an experiment description,
but most commonly they are not known for biological models and are
assumed to be one of the steady states of the model. A common setup is:
\begin{align}
  t_0&=-T&f(t,x(0;\rho,u);\rho,u)\overset{!}{=}0\,,
\end{align}
with $T$ being a large enough time to reach steady state and $x_0$
chosen suitably to hit the right steady state by whatever means.

\subsection{Stochastic Measurement Model}
\label{sec:out}

Since the measurements are noisy the model's output is compared to the
data using a statistical model. The right model is often unknown, in
fact it is a typical case that no uncertainty analysis has been
performed on the data at all and no repeated measurements have been
made to estimate the parameters of an error model hypothesis.

Until data sets come with carefully justified error models and
distribution parameters, we assume the error to be \emph{Gaussian}.

Since the software calculates gradients, with hard coded partial
derivatives, this choice is fairly static and cannot be changed
easily. A different error model requires the implementation of an
additional log-likelihood function and its partial derivatives. The
currently used functions are \emph{not supplied by the user}.

Similarly, since the Gaussian error model is merely an educated guess
that performs well numerically, we don't transform the distribution
into a possibly \emph{heavy tail}ed \emph{ratio distribution}.

We make the choice\footnote{rather than exactly transforming the
  distribution from $g$ to $h$ we match them by appropriate $\stdv{g}$
  transformation, but not in shape.} that the error is modeled as a
\emph{Gaussian} distribution at the highest level output function $h$
as well and we match that to the raw-distribution by appropriate
choice of $\stdv{h}$:
\begin{equation}
  \label{eq:error}
  \data{h}_{ijk} - h_i(t_j;\rho,u_k)\overset{!}{\sim}\mathcal{N}(0,\stdv{h}^2)\,,
\end{equation}
where $\stdv{h}$ denotes the possible transformation of $\stdv{g}$
if the standard deviation of the raw data\footnote{$\stdv{g}$ applies
  to the level-1 output $g_i(\cdot)$, $\stdv{h}$ applies to $h$.} $\stdv{g}$ has
  actually been estimated. The error propagation is calculated in the
  following way:
\begin{align}
  \label{eq:stdv_partial}
  \stdv{h}_{ijk}\approx&\left|\frac{\partial
      h_i(g_{ijk},g_{ij'0})}{\partial
      g_{ijk}}\right|\stdv{g}_{ijk} +
  \left|\frac{\partial h_i(g_{ijk},g_{ij'0})}{\partial
      g_{ij'0}}\right|\stdv{g}_{ij'0}\,,
\end{align}
and in this case:
\begin{align}
  \label{eq:stdv_ratio}
    \stdv{h}_{ijk}\approx&\left|\frac{h_i(g_{ijk},g_{ij'0})}{g_{ijk}}\right|\stdv{g}_{ijk} + \left|\frac{h_i(g_{ijk},g_{ij'0})}{g_{ij'0}}\right|\stdv{g}_{ij'0}\,.
\end{align}
Our goal is to provide an estimate\footnote{or perhaps even an upper
  bound} of uncertainty in the data and propagate this uncertainty to
the model's parameters. We consider these decisions to be a middle
ground between simplicity, numerical stability and accurate treatment
of measurements. The statistical model is probably not quite right,
but as it is almost always unknown we have nothing to replace it with.

Other sensible choices are to log-transform the data, such that an
assumption of log-normal errors is made. But in practice, the above
choices seem to be best compatible with biological data and
numerically stable whenever no error bounds have been reported.

The first layer output is defined as part of the model, inside the
model file. The characteristic of it is that one simulation at input
$u_k$ results in exactly one output $g_{ijk}$ ($n×\nu_k$ matrix). The
second layer, the normalisation of primary outputs:
$g_{ijk}/g_{i'j'k'}$ is to some extent fixed and only the indexing
rules (which $i',j',k'$) can be supplied via the annotation of the
data. The operation \texttt{ratio of g} is predefined and cannot be
changed easily (only turned off).

Log-transformations inside of $g$ are allowed already as that is
defined in the model itself. Log-transformations on the level of $h$
are not supported yet:
\begin{align}
  h_{ijk}&\overset{\texttt{OK}}{=} \log(x_i(t_j;\rho,u_k))/\log(x_i(t_{j'};\rho,u_0))\,,& g_{ijk}&=\log(x_i(t_j;\rho,u_k))\\
  h_{ijk}&\overset{\texttt{NO}}{=} \log(x_i(t_j;\rho,u_k)/x_i(t_{j'};\rho,u_0)))\,,&g_{ijk}&=x_i(t_j;\rho,u_k)\,.
\end{align}
This has to do with analytical, hard coded expressions during the
calculations of gradients and the Fisher information; not a principal
limitation.

\subsection{Sensitivities}
\label{sec:sens}

The model has sensitivities of $x$ with respect to $\rho$ and derived
from that also sensitivities of $g$ and $h$, with repsect to changes
in the parameters $\rho$ and on the log-scale: $\theta=\log(\rho)$.

The \software{cvodes} solver will return both the state sensitivities
$\sens{x}{i}{l}{k}$ and the output sensitivities
$\sens{g}{i}{l}{k}$. So, we will use these as given.

With a known normalisation indexing: $i',j',k'$ for each $i,j,k$, the
sensitivity of $h(\cdot)$ in terms of the (known) $g(\cdot)$
sensitivities $\sens{g}{i}{j}{k}$ is:
\begin{align}
    g_{ijk}(\rho):&=g_i(x(t_j;\rho,u_k))\,,\\
    h_{ijk}(\rho):&=\frac{g_{ijk}(\rho))}{g_{i'j'k'}(\rho)}\,,\\
  \frac{d h_i(g_{ijk}(\rho)),g_{i'j'k'}(\rho))}{d \rho_l}
  &= \frac{\sens{g}{i}{l}{k} g_{i'j'k'}(\rho) -g_{ijk}(\rho) \sens[j']{g}{i'}{l}{k'}}{g_{i'j'k'}(\rho)^2}\nonumber\\
  \sens{h}{i}{l}{k}&= \frac{\sens{g}{i}{l}{k} 
    - h_{ijk}(\rho)\sens[j']{g}{i'}{l}{k'}}{g_{i'j'k'}(\rho)}\,.\label{eq:fyS}
\end{align}
The code for this
operation is located in the function \texttt{LogLikelihood}.

\subsection{Sampling in Logarithmic Space}
\label{sec:rho}

In this type of \ode{} model, the parameters are positive. In some
cases the model becomes unstable after sign flips, so non-negativity
has to be enforced. Additionally, bio-chemical parameters are often
unknown even in their magnitude. So, sampling $\theta$ will result in
only positive values $\rho=\exp(\theta)$ to be passed to the model as
parameters.

This has the additional benefit of covering several orders of
magnitude more efficiently. Unfortunately, this choice implies that we
have to modify the expressions for the model sensitivities. The model
handling tool \software{vfgen} and solver \software{cvodes} provide
functions for sensitivity analysis with respect to the nominal model
parameters $\rho$:
\begin{equation}
  \label{eq:hsens_theta}
  \begin{split}
    \dot x&=f(t,x;\rho,u)\,,\\
    g_{ijk}(\rho):&=g_i(x(t_j;\rho,u_k))\,,\\
    h_{ijk}(\rho):&=\frac{g_i(x(t_j;\rho,u_k))}{g_{i'}(x(t_{j'};\rho,u_{k'}))}\,,\\
    \sens{h}{i}{l}{k}:&=\frac{d h_i(g_i(x(t_j;\rho,u_k)),g_{i'}(x(t_{j'},\rho,u_{k'})))}{d\rho_l}\,,\\
    \frac{dh_{ijk}(\rho(\theta))}{d\theta_l}&=\frac{\partial h_{ijk}(\rho(\theta))}{\partial\rho_l}\frac{\partial \rho_l}{\partial\theta_l}\\
    &=\sens{h}{i}{l}{k}\rho_l\,,
  \end{split}  
\end{equation}
for any input $u_k$ (no summation implied).

\section{Software Usage}
\label{sec:software}

The software has two major inputs: (i) the \ode{} model as a shared
library, and (ii) all data sets (annotated) as an \standard{hdf5} file.

\subsection{Model\hfill\texttt{.so}}
\label{sec:soModel}

If the user has a way to generate \software{cvodes} compatible model
source and header files (\texttt{C}) from some modeling language like
\standard{sbml} then nothing else is required. We provide no
\standard{sbml} to \texttt{C}/\software{cvodes} conversion scripts (currently).

One of the reasons is that to our knowledge the \standard{sbml}
standard does not provide any way to define model input parameters. In
addition, no standard software tools are known to us to process
\standard{sbml} into \texttt{C} sources automatically, and certainly
not including symbolically calculated Jacobians. Custom conventions
for the definition of inputs and outputs can be made, yet
\standard{sbml} is quite difficult to parse, while tabular formats are
easy to parse using line oriented tools. For all of these reasons we
have decided to use \standard{SBtab} for the editing and storage of
the model and all data sets (if small enough) for our projects.

The model can be converted from \standard{SBtab} using an \software{r}
script we provide:
\begin{center}
\texttt{sbtab\_to\_vfgen.R}\,,
\end{center}
as further explained in Section~\ref{sec:tools}.
It reads the \standard{SBtab} file in \emph{Open Document Spreadsheet}
(\texttt{.ods}) format and converts the biologically motivated model
into an ordinary dufferential equation model. This process strips
biological meaning somewhat (species,compartment,etc.) and uses only
general terms such as \emph{Expression} and \emph{State Variable}.

The result is a \software{vfgen} (\texttt{.vf/.xml}) file that can be
parsed by the
\software{vfgen}\footnote{\url{github.com/WarrenWeckesser/vfgen}}
software; \software{vfgen} in turn uses \software{ginac} to calculate
the \emph{Jacobian} and \emph{sensitivity} terms symbolically (for
this model). It outputs the model into a language of choice
(e.g. \textsc{matlab}, \textsc{R}) one of which is
\textsc{C/\{cvodes,gsl\}}.

The relevant commands are:
\begin{center}
\begin{tabular}{rl}
  \textbf{meaning}&\textbf{command in \texttt{bash}}\\
  vfgen to cvodes&\texttt{vfgen cvodes:sens=yes,func=yes ./C/Model\_cvs.c}\\
  make shared library&\texttt{gcc -shared -fPIC -o Model.so Model\_cvs.c}
\end{tabular}
\end{center}

The shared library is loaded by the sampler \texttt{ode\_smmala} and
can be produced in any other way as long as it follows the interface
requirements of \texttt{cvodes}.

As a workaround regarding solver failures on high dimensional problems
with forward sensitivity analysis there is a sensitivity approximation
routine that requires a \emph{parameter} Jacobian
$df(t,x;\rho,u)/d\rho$ to be available in the model \texttt{struct}.

\section{Data Storage\hfill\texttt{.h5}}
\label{sec:data}

The problem set up is given as an \standard{hdf5} \texttt{.h5} file. The
file shall contain annotated data sets, and prior probability density
parameters. The prior parameters can be $\mu_i,\stdv{g}_i$ for
independent Guassian distributions for each model parameter, or
$\mu,\stdv{g}$ for a multivariate Gaussian.

\subsection{Experiments}
\label{sec:exp}

It is not immediately clear what aggregate of information can be said
to belong to one \emph{experiment}. We have so far identified two
major types:
\begin{description}
\item[Dose Response] this experiment type consists of model responses
  to varying \emph{doses} of input. The data table consists of one or
  more columns of input and one or more columns of corresponding
  outputs. Each line of such a table requires a simulation of the
  model to opbtain the input/output relationship for a given
  parametrization. In such cases there can be only one measurement
  time to record an output. If more than one time was recorded, then
  we must treat the case as many\footnote{possibly very short} time
  series experiments.
\item[Time Series] here, a data table describes one simulation of the
  model and gives a record of an output measured at discrete
  timepoints. 
\end{description}

Both types of experiment have mandatory components aside from data
points: an estimate of standard deviations\footnote{for output
  function $i$, time point $j$, and input vector $k$} $\stdv{g}_{ijk}$,
which input to apply, time(s) to record outputs. As described in
earlier sections, the sampler will always assume a normal distribution
for output noise. While reading this file, \texttt{ode\_smmala} will transform $\stdv{g}$ into $\stdv{h}$, as described in Section~\ref{sec:out}.

The two types of experiment can be defined in the \standard{SBtab}
file if that is used. But, once the data is processed, it is stored in
blocks that represent a \emph{simultion unit}. So, time series data
remains unchanged, while dose responses are converted into one
\emph{data block} per line. The \standard{SBtab} file allows omission
of many values, especially data points. The \texttt{sbtab\_import}
processing of the spreadsheet will replace missing values with
defaults, the replacement can be inspected using standard
\standard{hdf5} tools like
\begin{center}
\texttt{h5dump -d /data/data\_block\_0 \emph{DataFile.h5}}\,.
\end{center}
The \texttt{h5} file may not omit columns or
rows\footnote{It is entirely untested what happens when NA or NaN
  values are passed to the sampler.}.

\subsection{Data File Structure}
\label{sec:h5structure}

The terminology of \standard{hdf5} files can be understood as somewhat
analogous to a zip archive with folders and files. In an \standard{hdf5}
file, the folders are called \emph{groups} and the files are
\emph{dataset}s. The datasets can have attributes. This is the
structure the sampler expects to find:
\begin{description}
\item[data~\texttt{GROUP}] a group that contains measured data. All
  \texttt{DATASET}s inside this group are considered data matrices.
  \begin{description}
  \item[data\_block\_\%i~\texttt{DATASET}] Consequtively numbered data
    sets with attributes. A matrix of size $n_\texttt{t}\times r$
    (\texttt{(size(t)*size(h))}). One line per measurement time point
    and one column per defined output. Missing values can be indicated
    by infinite standard deviations. The name \textsf{data\_block\_*}
    is not enforced or checked; the \textsf{index} attribute is used
    to order the data sets.
    \begin{description}
    \item[input~\texttt{ATTRIBUTE}] the input parameter vector for this data block
      (simulation unit).
    \item[time~\texttt{ATTRIBUTE}] the measurement times for this data
      block, one per row in the data matrix.
    \item[LikelihoodFlag~\texttt{ATTRIBUTE}] if this is $1$, the
      experiment will contribute to the likelihood function. If it is
      $0$, then it is assumed to be needed for the normalisation of
      another experiment. It will be simulated, can participate in
      normalisation operations but skipped during the calculations of:
      log-likelihood, its gradient and Fisher information.
    \item[NormaliseByExperiment~\texttt{ATTRIBUTE}] \emph{optional}
      Exactly one index that identifies which \texttt{dataset} this
      one is relative to. If this is missing or negative then the
      experiment is not relative to another. This corresponds to $k'$
      from Section~\ref{sec:out}.
    \item[NormaliseByOutput~\texttt{ATTRIBUTE}] \emph{optional} a vector of indices,
      one per output, to indicate which output to devide by. This
      corresponds to $i'$. If negative, then the output is absolute,
      the normalising value is set to $1\pm 0$.
    \item[NormaliseByTimePoint~\texttt{ATTRIBUTE}] \emph{optional} if present, then
      each line of this data block will be devided by exactly on line
      from the reference experiment. This corresponds to $j'$. This
      cannot be a vector. If this is missing, but
      \textsf{NormaliseByExperiment} is present, then the two must
      have the same number of lines. No check is performed whether the
      lines represent the same time points.
    \item[index~\texttt{ATTRIBUTE}] a running index, numbering the
      data blocks, starting at $0$. This index is used to find
      corresponding data blocks and standard deviation blocks (same
      index). There are two additional index numbers, major and
      minor. These can be used to check the correspondence between
      data blocks and \standard{SBtab} sheets, where \emph{dose
        response} sheets can contain several data blocks(simulation
      units).
    \item[major~\texttt{ATTRIBUTE}] an index that corresponds to data
      sheets, so all data blocks that come from the same \emph{dose
        response} experiment will share the same major index.
    \item[minor~\texttt{ATTRIBUTE}] an index that numbers the data entry lines
      from a \emph{dose reponse} experiment.
    \end{description}
  \end{description}
\item[stdv~\texttt{GROUP}] the group that contains the standard deviation estimates for the data points.
  \begin{description}
  \item[stdv\_block\_\%i~\texttt{DATASET}] a matrix of the same
    size as \textsf{data\_block\_\%i} (these are matched by the
    \textsf{index} attribute). Infinite standard deviations indicate a
    missing value. These are omitted from calculations such as
    $\prod_{ijk} \sqrt{2\pi}\stdv{g}_{ijk}$. The standard deviations
    don't repeat the required annotation of the data matrices,
    they only contain the running index and possibly major \& minor
    attribute.
    \begin{description}
        \item[index~\texttt{ATTRIBUTE}] a running index, numbering the
      data blocks, starting at $0$. This index is used to find
      corresponding data blocks and standard deviation blocks (same
      index).
    \item[major~\texttt{ATTRIBUTE}] an index that corresponds to data
      sheets, so all data blocks that come from the same \emph{dose
        response} experiment will share the same major index.
    \item[minor~\texttt{ATTRIBUTE}] an index that numbers the data entry lines
      from a \emph{dose reponse} experiment.
    \end{description}
  \end{description}
\item[prior~\texttt{GROUP}] Prior parameter group. The prior is
  either\footnote{not implemented yet: generalised Gaussian} a product
  of univariate \emph{Guassians} or one multivariate Gaussian. These
  two cases are distinguished by name: \textsf{sigma} refers to
  $\stdv{g}$, while \textsf{Sigma} refers to the multivariate case
  $\Sigma$ (a symmetric matrix). Select one of the optional items.
  \begin{description}
  \item[mu~\texttt{DATASET}] a vector of one $\mu$ per paramete
    $\theta$ (size $m$), in logarithmic space (natural logarithm).
  \item[sigma~\texttt{DATASET}] \emph{optional} prior parameter for
    \emph{width} (standard deviation), same size as \textsf{mu}.
  \item[Sigma~\texttt{DATASET}] \emph{optional} a matrix of size $m\times m$. The covariance matrix $\stdv{g}$ of the multivariate prior.
    \item[Precision~\texttt{DATASET}] \emph{optional} Same as $\Sigma^{-1}$.
  \end{description}
\end{description}

\subsection{Tools}
\label{sec:tools}

If the \standard{SBtab} method is used, then the \texttt{.ods} file can
be converted into a series of \texttt{.tsv} files and then converted
into an \standard{hdf5} file using the \texttt{sbtab\_import} program:
\begin{center}
  \begin{tabular}{rl}
    \standard{SBtab} to \standard{hdf5}&\texttt{sbtab\_import *.tsv DataSet.h5} 
  \end{tabular}
\end{center}
The git repository \url{github.com/a-kramer/SBtabVFGEN} contains
scripts for conversion of \standard{SBtab} files between the
\texttt{.tsv} and \texttt{.ods} formats and an
\software{r}\footnote{\href{https://www.r-project.org/}{r-project}}
script \texttt{sbtab\_to\_vfgen.R} that converts such a model into a \software{vfgen}
readable\footnote{also fairly human readable} \texttt{.xml} file.

An example model in the \standard{SBtab} form is provided here:
\url{github.com/a-kramer/DemoModel}.
\software{vfgen} can output the same model into many other languages for post processing of sampling results.
\end{document}
