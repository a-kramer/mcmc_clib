\documentclass[utf8,english]{scrartcl}
\usepackage{lmodern}
\usepackage[sc]{mathpazo}
\usepackage[T1]{fontenc}
\usepackage{inputenc}
\usepackage{babel}
\usepackage{amsmath,amssymb}
\usepackage{microtype}
\usepackage{tikz}


\newcommand{\ode}{\textsc{ode}}
\addtokomafont{title}{\rmfamily}
\addtokomafont{section}{\rmfamily}

\title{Documentation for the RMHMC Source}
\author{Andrei Kramer}
\date{\today}

\begin{document}
\maketitle
\begin{abstract}
  This text provides some documentation and mathematical
  considerations and definitions for the implementation of the
  Riemannian Manifold Hamiltonian Monte Carlo (RMHMC) algorithm.
\end{abstract}

\section{Model Specifications}
\label{sec:odeModel}

We consider a deterministic \ode{} model and a stochastic measurement
model. System states are captured by the state variables
$x(t)\in\mathbb{R}^n$. The model parameters $\theta\in\mathbb{R}^m$
describe interactions between the model state variables and are
unknown. A second set of parameters $u\in\mathbb{R}^l$ describes the
conditions of an experimental setup. These parameters are considered
inputs and we assume that they can be set by the experimenter. They
can be external parameters, e.g. the temperature, or describe
modifications to the system, e.g. inhibitions to some of the
interactions. The known time derivative of $x(t)$ defines the model:
\begin{align}
  \dot x &= f(t,x;\theta,u)\,,\\
  y(t_j;\theta,u)&=C(\zeta) x(t_j;\theta,u) + \epsilon_i(t_j)\,,
\end{align}
where $y(t;\theta,u)\in\mathbb{R}^k$ is the output of the measurement process,
which is recorded at $T$ time points. The linear transformation
$C\in\mathbb{R}^{k\times n}$ models the capabilities of the
measurement setup. The measurements are obscured by the noise process
$\epsilon_i(t_j)\sim\mathcal{N}(0,\sigma^2_{ij})$ ($i=1,\dots,n;
j=1,\dots,T$). In addition, the observations might be done in unknown
units, such that a reference Experiment might be needed to interpret
any numerical values of $y(t_j)$. To deal with this problem, $C$ may
contain unknown scaling parameters $\zeta$, to be estimated via MCMC. Another
approach, which instead eliminates the scaling parameters, is to take
the ratios:
\begin{align}
  \tilde y_i(t_j;\theta,u_b) &= \frac{y_i(t_j,\theta,u_b)}{y_i(t_j;\theta,u_0)}\,,
\end{align}
where $u_b\in\mathbb{R}^l (b=1,\dots,n_{\text{E}})$ is any particular
experimental setup ($n_{\text{E}}$ is the number of experiments) and
$u_0$ is the reference experiment setup.  In consequence we have to
recalculate the standard deviation of $\tilde y_i(t_j;\theta,u_b)$
from the standard deviations\footnote{we append the input indeces
  $b,0$ to the standard deviation symbol for distinction} of
${y_i(t_j,\theta,u_b)}$ and ${y_i(t_j;\theta,u_0)}$:
\begin{align}
  \label{eq:std_y}
  \tilde\sigma_{i,j}\approx&\left|\frac{\partial\tilde
      y_i(t_j;\theta,u_b)}{\partial
      y_i(t_j;\theta,u_b)}\right|\sigma_{ij,b} +
  \left|\frac{\partial\tilde y_i(t_j;\theta,u_b)}{\partial
      y_i(t_j;\theta,u_0)}\right|\sigma_{ij,0}\,,
\end{align}
and in this case:
\begin{align}
  \label{eq:std_yy}
  \tilde\sigma_{i,j}\approx&\left|\frac{\tilde
      y_i(t_j;\theta,u_b)}{y_i(t_j;\theta,u_b)}\right|\sigma_{ij,b} +
  \left|\frac{\tilde
      y_i(t_j;\theta,u_b)}{y_i(t_j;\theta,u_0)}\right|\sigma_{ij,0}\,.
\end{align}


\section{Sensitivities}
\label{sec:sens}

The sensitivity of $\tilde y(t)$ in terms of the (known) $y(t;\theta,u)$ sensitivities $S(t;\theta,u)$ is:
\begin{align}
  \partial_{\theta_j} \tilde y_i(t;\theta,u_b) 
  &= \frac{S_i^{~j}(t;\theta,u_b)y_i(t;\theta,u_0)
    -y_i(t;\theta,u_b)S_i^{~j}(t;\theta,u_0)}{y_i(t;\theta,u_0)^2}\nonumber\\
  &= \frac{S_i^{~j}(t;\theta,u_b) 
    - \tilde y_i(t;\theta,u_b)S_i^{~j}(t;\theta,u_0)}{y_i(t;\theta,u_0)}\,.  \label{eq:fyS}
\end{align}

The code for this operation is located in the function
\texttt{Likelihood} and is organized such, that if the data is
absolute and does not require normalization, then
$S_i^{~j}(t;\theta,u_0)$ is set to $0$ for all $i,j$ and the reference
$y_i(t;\theta,u_0)=1$ for all $i$.

\section{Sensitivity Gradient}
\label{sec:dS}

We take the derivative of~\eqref{eq:fyS} for any $u_b\in\{u_1,\dots,u_{n_{\text{E}}}\}$:
\begin{multline}
%  \label{eq:dfyS}
  \partial_{\theta_k} \tilde S_i^{~j}(t;\theta,u_b)
  % &=\frac{\left(\frac{\partial S_i^{~j}(t;\theta,u_b)}{\partial
  %   \theta_k} - \left(\tilde
  %     S_i^{~j}(t;\theta,u_b)S_i^{~j}(t;\theta,u_0) + \tilde
  %     y_i(t;\theta,u_b)\frac{\partial
  %     S_i^{~j}(t;\theta,u_0)}{\partial
  %     \theta_k}\right)\right)y_i(t;\theta,u_0) -
  % \left(S_i^{~j}(t;\theta,u_b) - \tilde
  %   y_i(t;\theta,u_b)S_i^{~j}(t;\theta,u_0)\right)S_i^{~k}(t;\theta,u_0)}{(y_i(t;\theta,u_0))^2}\\[2mm]
  =\left(\frac{\partial S_i^{~j}(t;\theta,u_b)}{\partial\theta_k} -\right.
    \tilde S_i^{~k}(t;\theta,u_b)S_i^{~j}(t;\theta,u_0) -\\ 
    \left.\tilde y_i(t;\theta,u_b)\frac{\partial S_i^{~j}(t;\theta,u_0)}{\partial
      \theta_k}\right) \frac{1}{y_i(t;\theta,u_0)}\\
  +\left(\tilde y_i(t;\theta,u_b) S_i^{~j}(t;\theta,u_0) - S_i^{~j}(t;\theta,u_b)\right)\frac{S_i^{~k}(t;\theta,u_0)}{(y_i(t;\theta,u_0))^2}\label{eq:dfyS}\,.
\end{multline}

\section{Sampling in logarithmic Space}
\label{sec:rho}

Ode models are often unstable for negative parameters, which is
definitely the case for biological models.  Sampling in logarithmic
space $\theta$ and passing $\rho=\exp(\theta)$ to the ode system fixes
the problem. But it also means that we have to modify the above
expressions. Because \texttt{vfgen} and \texttt{cvodes} provide
sensitivity analysis with respect to the nominal model parameters
$\rho$:
\begin{align}
  \dot x&=f(t,x;\rho,w)\,,& \rho_j&=\exp(\theta_j)\,,\nonumber\\
  y(t;\rho,w)&=C x(t;\rho,w)\,,& b&=1,\dots,l\,,\nonumber\\
  \tilde y_i(t;\rho,w)&=\frac{y_i(t;\rho,w)}{y_i(t,\rho,u_0)}\,,\label{eq:ode_rho}\\
  \tilde S_i^{~j}(t;\rho,w)&=\partial_{\rho_j} \tilde y_i(t;\rho,w)\,,\nonumber\\
  \partial_{\theta_j}\tilde y(t;\rho,w)&=\frac{\partial\tilde y_i(t;\rho,w)}{\partial\rho_j}\frac{\partial \rho_j}{\partial\theta_j}\nonumber\\
  &=\tilde S_i^{~j}(t;\rho,w)\rho_j\,.\nonumber
\end{align}
for any input $w$ and consequently:
\begin{align}
  \partial_{\theta_j}\tilde y_i(t;\rho,w)&=\tilde S_i^{~j}(t;\rho,w)\rho_j\,,\nonumber\\
  \partial_{\theta_k}\tilde S_i^{~j}(t;\rho,w)\rho_j&=\frac{\partial
    \tilde
    S_i^{~j}(t;\rho,w)}{\partial\rho_l}\overbrace{\frac{\partial\rho_l}{\partial\theta_k}}^{\rho_l\delta_{lk}}\rho_j
  + \tilde
  S_i^{~j}(t;\rho,w)\frac{\partial\rho_j}{\partial\theta_k}\label{eq:dfyS_rho}\\
  &=\frac{\partial \tilde
    S_i^{~j}(t;\rho,w)}{\partial\rho_k}\rho_k\rho_j + \tilde
  S_i^{~j}(t;\rho,w)\rho_k\delta_{jk}\,.\nonumber
\end{align}
Note that it is possible to use \texttt{Expression}s in vfgen models:
\begin{center}
\begin{verbatim}
<Parameter Name="\theta_1" DefaultValue="0">
<Expression Name="rho_1" Formula="exp(theta_1)"/>
\end{verbatim}
\end{center}
to convert the parameters from logspace, then use these expressions to
define fluxes. But, since \texttt{Expression}s need to be recalculated
at every step this is very wasteful. On the other hand, \textsc{vfgen}
will then compute correct sensitivities ($dx_i/d\theta_j$). But, to
save calls to the exp function, the software does this conversion
before calling the solver and instead transforms the sensitivities as
described here.
\end{document}
