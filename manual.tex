\documentclass[english,12pt]{scrartcl}
\usepackage{lmodern}
\usepackage[osf]{mathpazo}
\usepackage[T1]{fontenc}
\usepackage[utf8]{inputenc}
\usepackage{microtype}
\usepackage{babel}
\usepackage{amsmath,amssymb}
\usepackage[bibstyle=numeric,backend=bibtex]{biblatex}
\usepackage{varioref}
\usepackage{array} % for tables with <{} and >{} column modifiers
\usepackage{booktabs}
\usepackage{capt-of}
\usepackage{multicol}
\usepackage{listingsutf8}
\lstset{language=bash,
        inputencoding=utf8,
        extendedchars=true,literate={«}{<<}1 {»}{>>}1,
        basicstyle=\ttfamily\small, 
        breaklines=true}
\usepackage[usenames,dvipsnames]{xcolor}
% some color definitions
\colorlet{lcolor}{blue!40!black}
\colorlet{ucolor}{magenta!40!black}
\colorlet{ccolor}{green!40!black}
\usepackage[colorlinks=true,%
            linkcolor=lcolor,%
            urlcolor=ucolor,%
            citecolor=ccolor]{hyperref} % makes references clickable
% style decisions
\setcapindent{0em}
\addtokomafont{captionlabel}{\bfseries}
\addtokomafont{title}{\rmfamily}
\addtokomafont{sectioning}{\rmfamily}
\frenchspacing


%bibliography
\addbibresource{NumericalSoftware.bib}


% makros
\newcommand{\CLIB}{\textsc{mcmc\_clib}}

% titling
\title{MCMC\_CLIB User Manual}
\author{Andrei Kramer}
\date{\today}

\begin{document}
\maketitle
\begin{abstract}
  This manual lists how to set up an ODE model for parameter sampling
  using the \CLIB{} software package. We have provided very general
  example files, which illustrate all uses of the software package. In
  the provided example the data is given in arbitrary units and only
  makes sense in relation to a reference experiment. Since the
  reference data also depends on the unknown parameters, the ratio has
  to be taken every time the model is used in the likelihood
  function. The example shows how to model unknown but stable initial
  conditions of perturbation experiments. Note that this manual does
  not provide documentation on the mathematical structure of the
  sampler. 
\end{abstract}
\tableofcontents
\section{Setup}
\label{sec:setup}

This section provides all necessary commands to generate a shared
library which can be loaded by \CLIB{}. The initial value problem for
the given ODE model will be solved using CVODES. We use
\textsc{vfgen}~\cite{vfgen} for this purpose.

\subsection[make model]{Make Model\hfill\texttt{model.vf}}
\label{sec:model}

\textsc{Vfgen} takes an \texttt{xml} file as input and exports into
many different formats including \textsc{cvodes} and
\textsc{matlab}. The extension \texttt{.vf} is arbitrary.

We use the xml-element \texttt{Parameter} to define model parameters
$\theta$ and inputs $u$. This list of parameters is ordered and the
\CLIB{} software will assume that the first $m$ appearing parameters
are the unknown sampling parameters $\theta$, where $m$ is defined
implicitely by setting a prior density in the \CLIB{} configuration
file, see Section~\vref{sec:configuration}. The remaining parameters
will be used as known input parameters. These inputs model the (known)
experimental conditions, they differ from one experiment to another.

To model output functions, we use the \texttt{Function} element, while
fluxes can be defined using the \texttt{Expression} element.

\subsection[make \textsc{cvodes} model sources]{Make \textsc{cvodes}
  model sources\hfill\texttt{model\_cvs.\{c,h\}}}
\label{sec:c}

Export the model:
\begin{verbatim}
$ vfgen cvodes:sens=yes,func=yes model.vf
\end{verbatim}

\subsection[make a shared library]{Make a Shared Library\hfill\texttt{model.so}}
\label{sec:so}

Compile the \texttt{vfgen} sources with \textsc{gcc}:
\begin{verbatim}
$ gcc -shared -fPIC -O2 -Wall -o model.so  model_cvs.c
\end{verbatim}

\subsection[write a configuration/data file]{Write a Configuration File\hfill\texttt{data.cfg}}
\label{sec:configuration}

The configuration file is a plain text file; it contains some
mandatory and some optional elements. Optional elements can be set for
convenience, the same elements can be set using command line
arguments. Command line arguments take precedence over the entries in
the configuration file.

The mandatory elements concern the data, the experimental conditions
under which the data was recorded and the prior knowledge about the
parameters. Most of the mandatory elements can be described as blocks
of numerical entries or matrices. This type of entry is done using
tags of the form \texttt{[entry\_type][/entry\_type]}:
\begin{description}
\item[\texttt{[time]}] A row of timepoints at which measurements have
  occured. This array has a meaning for all state variables. If some
  outputs were not observed at some time instance, this is noted in
  the \texttt{data} tag. The closing tag is \texttt{[/time]}.
\item[\texttt{[reference\_input]}] This block contains a row of known
  input parameters which define the model conditions corresponding to
  the reference experiment. If this block is present, then the
  experiment is considered relative. The closing tag is
  \texttt{[/reference\_input]} and similar for the following.
\item[\texttt{[input]}] contains one row per experiment, with columns
  corresponding to the input parameters.
\item[\texttt{[reference\_data]}] block containing the measurements
  for the reference experiment; has one row per time point. columns
  represent output functions.
\item[\texttt{[data]}] The data block can take data from any number of
  performed experiments. Each experiment is associated with an input
  vector. Columns represent the output functions defined in the
  \texttt{vf} file. Rows represent measurement time points and
  experiment conditions (see Table~\ref{tab:datastrcuture}):
  measurements observed at timepoint $t_i$ ($i=0,\dots,T-1$) for input
  $u_j$ ($j=0,\dots,N-1$) will be in row $k=jT+i$
  ($k=0,\dots,NT-1$). For example, with $T$ time points and $N$
  experimental conditions, we get:
  \begin{multicols}{2}
  \begin{tabular}{>{$}c<{$} >{$}r<{$} >{$}l<{$}}
    \toprule
    \text{line}&\text{input row}&\text{timepoint}\\
    \midrule
    1&1& 1 \\
    2&1& 2 \\
    \vdots&1& \vdots \\
    &1& T\\
    &2& 1 \\
    &2& 2 \\
    &2& \vdots \\
    &2& T\\
    &\vdots&\vdots\\
    NT&N&T\\
    \bottomrule
  \end{tabular}
  \columnbreak
  \captionof{table}{Here we used line numbering starting at $1$; we get $NT$ lines
    for $N$ inputs (experiments) and $T$ timepoints at which
    measurements have occurred. Each line contains a row of data
    points. If a measurement was omitted you can enter any placeholder
    in the omitted slot and set the standard diviation of this point
    to \texttt{inf} ($\infty$).\label{tab:datastrcuture}}
  \end{multicols}
\item[\texttt{[sd\_data]}] This block is of the same size and
  structure as the data block and contains the standard deviations
  (uncertainties) of the data. Enter \texttt{inf} for any omitted
  measurement.
\item[\texttt{[prior\_mu]}] The prior is assumed to be log-normal on
  all unknown model parameters. Since sampling is done in logarithmic
  space, the prior is a multivariate normal $\mathcal{N}(\mu,\Sigma)$
  for the actual sampling variables. This block contains one column;
  it represents $\mu$ and has one row per unknown parameter. It is
  very important to set $\mu$ to values that make sense in
  \emph{logarithmic} space.
\item[\texttt{[prior\_inverse\_cov]}] This tag contains the $\Sigma$
  parameter of the Gaussian prior. It can also be written like this:
  \texttt{[prior\_inverse\_covariance]}.
\end{description}
For convenience, you can set some of the command line options also as
variables inside the configuration file. Some variables can only be
set here. The follwong table lists all variables with example values:
\begin{center}
\begin{tabular}{r@{\texttt{=}}ll}
\toprule
variable name&value&meaning\\
\midrule
\texttt{step\_size}&\texttt{0.001}&Sampling algorithm's initial step size\\
\texttt{sample\_size}&\texttt{1000000}&\texttt{Integer}, sample size after warmup\\
\texttt{acceptance}&\texttt{0.50}&Target acceptance for tuning. \\
\texttt{output}&\texttt{./sample/model\_Y.double}&sample will be writen to this file.\\
\texttt{t0}&\texttt{-72}&$t_0$ for the initial conditions $x(t_0)=x_0$.\\
\bottomrule
\end{tabular}
\end{center}
Setting $t_0$ to a large negative value can be used to model
experiments which start in a stable but unknown equilibrium state of
the model. While the input function inside the model (\texttt{vf}
file) can be set to perform a perturbation at $t=0$.


\section{Run Simulation}
\label{sim}

The program has some command line options which can be shown by typing:
\begin{verbatim}
$ ./ode_smmala -h
Usage: -c ./data.cfg
                        data.cfg contains the data points and the
                        conditions of measurement.

-l ./ode_model.so
                        ode_model.so is a shared library containing
                        the CVODE functions of the model.

-s $N
                        $N sample size. default N=10.

-o ./output_file
                        file for output. Output can be binary.

-b
                        output mode: binary.

-a $a
                        target acceptance value (markov chain will
                        be tuned for this acceptance).

--seed $seed
                set the gsl pseudo random number generator seed
                to $seed.
\end{verbatim}
An example of how to run the program is included in the package:
\texttt{run\_test.sh}.
\lstinputlisting{./run_test.sh}

\section{Sample Analysis}
\label{sec:analysis}

The output will be logarithmic and will contain the log posterior for
each parameter vector. The sample can be read and processed using
numerical computation software like \textsc{GNU Octave}~\cite{octave:2012} or MATLAB:
\begin{verbatim}
fid=fopen('sample.double','r');
Sample=fread(fid,[NumberOfParameters+1,SampleSize],'double');
fclose(fid);
\end{verbatim}
To obtain model trajectories the parameter values have to be
transformed:
\begin{verbatim}
LogPosteriorIndex=NumberOfParameters+1;
[~,Imax]=max(Sample(LogPosteriorIndex,:));
MaxLikelihoodParas=exp(Sample(1:NumberOfParameters,Imax));
\end{verbatim}

\printbibliography
\end{document}
